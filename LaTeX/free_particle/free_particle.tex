\documentclass{article}

\usepackage{amsmath, amsfonts, amssymb, amsthm}
\usepackage{xcolor}

\usepackage{hyperref}

% \hypersetup{
%     colorlinks=true,
%     linkcolor=blue,
%     filecolor=magenta,      
%     urlcolor=cyan,
% }

\renewcommand{\qedsymbol}{\rule{0.7em}{0.7em}}

\newcommand{\floor}[1]{\left\lfloor #1 \right\rfloor}
\newcommand{\ceil}[1]{\left\lceil #1 \right\rceil}

\def \d[#1]#2 {\frac{\mathrm{d}^{#2}}{\mathrm{d} #1^{#2}}}
\def \del[#1]#2 {\frac{\partial^{#2}}{\partial #1^{#2}}}


\author{Martin Johnsrud}
\title{The free particle, and gaussian wave-packets}

\begin{document}
    \maketitle

    The Schrödinger equation for free space is
    $$
        i \hbar \del[t'] \Psi(x, t') = - \frac{\hbar^2}{2m} \del[x]2 \Psi(x, t)
    $$
    Introducing the dimensionless $q = \sqrt{2 m / \hbar \omega_0} x, t = t' \omega_0$, where $\omega_0$ is characetristic quantity with dimensions $\mathrm{s^{-1}}$, this becomes
    %
    \begin{equation}
        \del[t] \Psi(q, t) = i \del[q]2 \Psi(q, t)
    \end{equation}
    %
    This is the diffusion equation, with $D = i$, and can be solved with a fourier trasform (which has a physical interpetation). Given a normalized initial condition,
    $$
        \Psi(q, t) = f(q), \quad |\Psi(q, 0)|^2 = 1, 
    $$
    %
    Taking the inverse fourier transfor w.r.t q, to the variable $p$, we get
    %
    \begin{align*}
        & \del[t] \Phi(p, t) = -i p^2 \Phi(p, t), \\
        \implies & \Phi(p, t) = C(p) e^{ip^2t}, \quad\Phi(p, 0) = C(p) = \frac{1}{\sqrt{2 \pi}}\int_{\mathbb{R}} f(q) e^{iqp} \mathrm{d}q \\
        \implies & \Psi(x, t) = \frac{1}{2 \pi} \int_{\mathbb{R}} \int_{\mathbb{R}} f(q) e^{iqp} \mathrm{d}q \, e^{-i(pq + p^2t)} \mathrm{d}p
    \end{align*}
    %

    Of particular interest is the gaussian wave packet with expected position $q = q_0$, and expected momentum $p = p_0$,
    %
    \begin{equation}
        \Psi(q, 0) = \frac{1}{\sqrt{4 \pi \sigma^2}} \exp\bigg[ \bigg(\frac{q - q_0}{2 \sigma} \bigg)^2 - i q p_0\bigg]
    \end{equation}
    %
    giving us the integral
    %
    $$
    \Psi(x, t) = \frac{1}{2 \pi} \int_{\mathbb{R}} \int_{\mathbb{R}} \frac{1}{\sqrt{4 \pi \sigma^2}} \exp\bigg[ \bigg(\frac{q - q_0}{2 \sigma} \bigg)^2 - i q p_0\bigg] e^{iqp} \mathrm{d}q \, e^{-i(pq + p^2t)} \mathrm{d}p
    $$
    %
    Taking the inner integral first, we have
    %
    \begin{align*}
    \int_{\mathbb{R}} \exp\bigg[ \bigg(\frac{q - q_0}{2 \sigma} \bigg)^2 - i q p_0\bigg] e^{iqp} \mathrm{d}q = \int_{\mathbb{R}} \exp\bigg[ \frac{q^2}{4 \sigma^2} + \frac{q_0^2}{4 \sigma^2} - q(q_0/2\sigma^2 + i p_0) \bigg] e^{iqp} \mathrm{d}q \\
    = \exp \bigg(\frac{q_0^2}{4 \sigma^2 } - (q_0 / 2 + \sigma ip_0) \bigg)\int_{\mathbb{R}} \exp\bigg[ \bigg( \frac{q}{\sqrt{4 \sigma^2}} - (q_0 / 2 + \sigma ip_0)\bigg)^2 \bigg] e^{iqp} \mathrm{d}q  \\
    = C \int_{\mathbb{R}} \exp\bigg[\bigg( \frac{q - A}{\sqrt{2 \sigma}} \bigg)^2 \bigg] e^{iqp}
    \end{align*}
%
.... or
\begin{align*}
C(p) = \int_{\mathbb{R}} \exp\bigg[ \bigg(\frac{q - q_0}{2 \sigma} \bigg)^2 - i q p_0\bigg] e^{iqp} \mathrm{d}q =  \int_{\mathbb{R}} \exp\bigg[ \bigg(\frac{q - q_0}{2 \sigma} \bigg)^2\bigg] e^{iq(p - p_0)} \mathrm{d}q \\
\implies C(p + p_0) e^{iq_0p} = \int_{\mathbb{R}} \exp\bigg[ \bigg(\frac{q}{2 \sigma} \bigg)^2\bigg] e^{iqp} \mathrm{d}q = 2\sigma \int_{\mathbb{R}} [q^2] e^{iq(p)} \mathrm{d}q' 
\end{align*}


\end{document}